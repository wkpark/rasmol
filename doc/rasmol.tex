% RasMol V2.10 Manual
% (c) Roger Sayle, May 1992
% Biocomputing Research Unit
% University of Edinburgh
%

\documentstyle[12pt,a4,lscape,makeidx]{article}
\makeindex

\newcommand{\up}[1]{\,$^{\underline{\hbox{\scriptsize\rm #1}}}$}
\setlength{\parindent}{0pt}
\setlength{\parskip}{2ex}

\begin{document}

\begin{center}
\mbox{}\vspace{3in} \\
{\huge\bf RasMol V2.2} \\
\vspace{2ex}
{\Large\bf A Molecular Visualisation Program} \\
\vspace{3ex}
\large
Roger Sayle \\
Biocomputing Research Unit \\
University of Edinburgh \\
\vspace{3ex}
\normalsize
last updated 14\up{th} October 1993
\end{center}
\newpage

% A Touch of Humour!
\index{Arguments!Circular|see{Circular Arguments}}
\index{Circular Arguments|see{Arguments, Circular}}

\small
\mbox{}\vfill
Copyright \copyright{} 1992,1993 Roger Sayle. \\
All Rights Reserved.

The information supplied in this document is believed to be true but
no liability is assumed for its use or for the infringements of the
rights of the others resulting from its use.

Information in this document is subject to change without notice and does
not represent a commitment on the part of the supplier. This package is
sold/distributed subject to the condition that it shall not, by way of
trade or otherwise, be lent, re-sold, hired out or otherwise circulated
without the supplier's prior consent in any form of packaging or cover
other than that in which it was produced. No part of this manual or
accompanying software may be reproduced, stored in a retrieval system on
optical or magnetic disk, tape or any other medium, or transmitted in any
form or by any means, electronic, mechanical, photocopying, recording or
otherwise for any purpose other than the purchaser's personal use.

This product is not to be used in the planning, construction, maintenance,
operation or use of any nuclear facility nor the flight, navigation or
communication of aircraft or ground support equipment. The author shall
not be liable, in whole or in part, for any claims or damages arising
from such use, including death, bancruptcy or outbreak of war.

\footnotesize
The X Window System is a trademark of the Massachusetts Institute of 
Technology \\ Alchemy is a trademark of Tripos Associates, Inc. \\
PostScript is a registered trademark of Adobe Systems Incorporated. \\
The Graphics Interchange Format\copyright{} is the Copyright property of
CompuServe Incorporated. \\ 
GIF is a Service Mark property of CompuServe Incorporated.
\normalsize
\newpage

\tableofcontents\clearpage

\section{Introduction}
\index{Introduction}
\label{Introduction}

RasMol is an menu- and command-driven interactive utility for the 
visualisation of molecular structures. The program will run on any BSD-like 
UNIX system, optionally displaying frames to either an 8 bit or 24 bit 
colour frame buffer available over the X Window System. Support is also 
provided for a dials box (via the XInput extensions) if one is connected 
to the system. 

RasMol v2.1 accepts atomic co-ordinates in the standard Brookhaven Protein 
Data Bank (PDB) format, optionally supplemented David Bacon's Raster3D 
colour scheme specifications. The program interactively displays the
molecule on the screen in a variety of representations, including wireframe,
backbone, union-of-spheres, sticks and ball-and-stick. Different portions
of the molecule may be rendered in any representation and colour independent
of the rest of the model. The molecule may be rotated, translated, zoomed
and z-clipped interactively, either from the keyboard, scroll bars or dials
attached to the workstation. Finally, the rendered image may be written to
a file in a variety of file formats.
\index{RasMol}

\section{General Operation}
\index{General Operation}
\label{General}

\subsection{Running RasMol}
\index{Running RasMol}
\label{Running}
To invoke the RasMol molecular visualisation tool, the user should type
the command `{\tt rasmol}' from the unix prompt. Immediately upon starting,
the program displays the following message to identify the version number
of the running program:

\begin{verbatim}
        RasMol Molecular Renderer
        Roger Sayle, June 1991
        Version 2.1B
\end{verbatim}

Immediately underneath this banner message, appears the program's command
line prompt `{\tt RasMol>~}'. If the program is being executed under the
X Window System, the program determines the type of the display being used.
If the screen has either an 8 bit or 24 bit colour frame buffer, RasMol
creates another window, which is used to display menu options and the
rendered images. If a suitable screen is not available, RasMol may only be
used from the command line. Commands may be typed to manipulate the model,
and to output the generated image to a raster file.
\index{X Windows|see{X Window System}}
\index{X Window System}
\index{Raster files}

If an optional argument is given on the unix command line, this is taken
to be a Brookhaven Protein Data Bank (PDB) file which is automatically
loaded by the program (see the {\tt load} command). If the file is found,
the program displays the usual statistics after reading in the file,
otherwise the error message `{\tt Error: File not found!}' is displayed,
before the user is presented the RasMol prompt.
\index{Brookhaven Protein Data Bank|see{PDB files}}
\index{PDB files}


In order to leave RasMol, the user can type the command {\tt quit} or 
{\tt exit} at the RasMol prompt, and the program will return the user to
the familiar unix prompt. Alternatively, if a prompt other than the main
RasMol prompt is being displayed, the user may hit control-C (\verb"^C")
to leave the program. The message `{\tt *** Quit ***}' will be output to
the terminal, before the usual unix prompt is redisplayed. The program may
also be terminated by selecting the {\bf Quit} menu option, on the bottom
of the main menu.
\index{Quit}
\index{Exit}


\subsection{Display Window}
\index{Display Window}
\label{Window}
If the program is run under the X Window System environment with a suitable
colour screen, RasMol creates an additional window to display the rendered
molecule interactively, as it is manipulated. This window is subdivided into
three main regions. The left hand side of the window, called the `canvas',
is used to draw the images of the molecule. By default this area initially
displays a black background.  Both below and to the right of the canvas are
two scroll bars used to rotate the molecule interactively. Finally, to the
far right of the window are the buttons that form the menu.
\index{X Window System}
\index{Canvas}
\index{Scroll bars}
\index{Menu}

While the mouse pointer is located within the canvas area of the display
window, the mouse pointer is drawn as a cross-hair cursor, to enable the
`picking' of objects being displayed (see later); otherwise the mouse pointer
is drawn as an arrowhead. Any characters that are typed at the keyboard
while the display window is in `focus' are redirected to the command line
in the terminal window.

The display window may be resized at any point during the session. This has
the effect of simply rescaling the image displayed on the canvas. RasMol
imposes limits on the size of the display window such that the window must
be large enough to display the menu and scroll bars and yet small enough to 
fit on a single screen. Attempts to enlarge the screen may fail owing to
insufficient memory on the host machine, in which case RasMol reports the
error message `{\tt Renderer Error: Unable to allocate frame buffer!}' or
some similar error.
\index{Display Window!size of}

On eight bit displays, when the number of colours required by the program
exceeds the number of free colours on the screen, the program uses its
own colourmap. This has the effect of temporarily displaying all windows
other than the display window in false colours while the mouse pointer is
within the display windows. If the mouse pointer is moved outside the
display windows, the original colours of the other windows return, and the
image on the canvas is shown in `false colour'. Once the number of colours
required by the program drops again, the presentation of colours returns
to normal.

\subsection{Scroll Bars}
\index{Scroll bars}
\label{Scroll}
The scroll bar across the bottom of the canvas area is used to rotate the
molecule about the y-axis, i.e. to spin the nearest point on the molecule
left or right; and the scroll bar to the right of the canvas rotates the
molecule about the x-axis, i.e. the nearest point up or down. Each scroll
bar has a `indicator' to denote the relative orientation of the molecule,
which is initially positioned in the centre of the scroll bar. These scroll
bars may be operated in either of two ways. The first is by clicking any mouse
button on the dotted scroll bar background to indicate a direct rotation
relative to the current indictor position; the second is by clicking one of
the arrows at either end of the scroll bar to rotate the molecule in fixed
sized increments. Rotating the molecule by the second method may cause the
indicators on the scroll bars to wrap around from one end of the bar to the
other. A complete revolution is indicated by the indicator travelling the 
length of the scroll bar. The angle rotated by using the arrows depends
upon the current size of the display window. No facility is provided for
depressing the mouse button and `dragging' the indicator along the scroll 
bar.
\index{Rotating the molecule}

\subsection{Menu Buttons}
\index{Menu buttons}
\label{Menu}
The menu consists of a vertical column of large buttons that are selected
by depressing and releasing any of the mouse buttons while the mouse
pointer is over a menu button. Each menu option either directly changes
the system options, or displays a submenu from which the user can select a
further option. The number of menu buttons displayed on the screen varies 
from menu to menu. Most menus have the option {\bf Cancel} to allow the 
user to return to the menu without affecting any of the current options. 
When selection of a menu option causes the program to prompt the user for 
additional information in the terminal window, the current command line 
being edited is erased (see later). When RasMol first starts executing,
the menu area of the display window presents the main menu. The bottom
option from this menu, {\bf Quit}, allows the user to terminate the 
program directly, and return to the UNIX prompt.
\index{Menu}
\index{Submenu}
\index{Cancel}
\index{Main Menu}
\index{Quit}

\subsection{Picking}
\index{Picking}
\label{Picking}
In order to identify a particular atom or bond being displayed, RasMol
allows the users to `pick' objects on the screen. The mouse is used to
position the cross-hair cursor over the appropriate item, and then any of
of the mouse button is depressed. Provided that the pointer is located 
close enough to a visible object, the program determines the identity of 
the nearest atom to the point identified.
\index{Atom!indentification of}

The program will display, in the terminal window, the atom's type, serial
number, residue name and residue number. If the atom is a member of a
named chain, the chain identifier is also displayed. Two examples of the
output generated by selecting an atom are displayed below:
\index{Residue type}
\index{Residue number}
\index{Chain identifier}

\begin{verbatim}
        Atom: CA 349  Group: SER 70
        Atom: O  526  Hetero: HOH 205  Chain: P
\end{verbatim}

The first line describes the alpha carbon of the serine-70 amino acid in
a protein. The unique Brookhaven serial number for this atom is 349. The
following line describes the oxygen atom in a water molecule attached to the
P chain of the main molecule. The word `Hetero' distinguishes heterogenous
molecules (such as cofactors) from the residues in the main molecule, noted
by `Group'. [These two atoms are referred to by the two atom expressions 
``{\tt SER70.CA}'' and ``{\tt HOH205P.O}'', respectively, when using the 
RasMol commands {\tt select} and {\tt restrict}.]
\index{Atom expression}

\subsection{Dials Box}
\index{Dials box}
\label{Dials}
If RasMol detects a `dials box' attached to the user's workstation, it
also allows the molecule to be manipulated interactively by the dials.
Once RasMol starts up, it labels the LED displays above each dial,
``ROTATE~X'', ``ROTATE~Y'', ``ROTATE~Z'' and ``ZOOM'' across the top
row from left to right, and ``TRANS~X'', ``TRANS~Y'', ``TRANS~Z'' and
``SLAB'' from left to right across the bottom row. Rotating any of the
knobs will automatically transform and redisplay the molecule 
interactively. The dials only have effect while the mouse pointer is
within the display window. If more than one application is using the
dials box at a time, care must be taken to remember the dial labels
assigned by each program, as each application may overwrite the 
dial-label LEDs.
\index{Rotating the molecule}
\index{Rotate!dial}
\index{Trans!dial}
\index{Zoom!dial}
\index{Slab!dial}

The rotation about the X and Y axes automatically updates the indicators 
on the appropriate scroll bars. All the rotation dials rotate the molecule 
$180^{\mbox{o}}$ for a complete revolution of the dial. All the remaining 
dials clamp their values to permissible ranges; turning these dials past 
their limits has no effect. 
\index{Scroll bars}

The ``ZOOM'' dial allows the interactive zooming of the molecule between 
10\% and 200\% of the original default magnification. Rotating the dial 
clockwise magnifies the molecule and anticlockwise shrinks it. A complete 
revolution of the dial corresponds to a 100\% change in scale.
\index{Zoom!dial}

The ``SLAB'' dial, which is only effective when slabbing is enabled, allows
the user to move the front z-clipping plane from the nearest point on the
molecule to the furthest. A complete rotation of the SLAB dial corresponds
to moving the clipping plane half the distance between the front and back of
the molecule. Turning the SLAB knob clockwise moves the clipping plane
closer to the viewer (increasing the number of objects displayed), and turning
it anticlockwise moves it further away (preventing more objects from being
displayed). Slabbing mode is enabled by selecting the {\bf Slab} menu button 
on the {\bf Options} menu, or by using the {\tt slab} command on the command 
line. 
\index{Options submenu}
\index{Slab!dial}
\index{Slab}

The translation along the X and Y axis allows the centre of the molecule 
to be moved within the canvas area of the screen. Rotation and zooming are 
still performed relative to the centre of the molecule, which may often not 
be at the centre of the canvas. The TRANS Z dial currently has no effect.
\index{Trans!dial}

\subsection{Command Line Interface}
\index{Command Line Interface}
\label{Command}
RasMol allows the execution of interactive commands typed at the RasMol
prompt in the terminal window. Characters typed into either the terminal 
or the display window are processed on the command line. Each command must
be given on a separate line terminated by a newline or carriage return
character. Keywords are case insensitive and may be entered in both lower
and upper case letters. All whitespace (space, tab and formfeed) characters
are ignored, except to separate the keyword and the arguments of a command. 
Blank lines (those containing only whitespace) are ignored. There is an
internal restriction that command lines are limited to a maximum of 256
characters. Strings may be delimited by matching single or double quotation 
marks. Placing a hash `{\tt \#}' character anywhere outside quotes 
terminates the line. RasMol will ignore the rest of the line, which may 
be used to comment on the command.
\index{Commands}
\index{Strings}
\index{Quotation marks}
\index{Hash Character}
\index{Comments}

If a syntax error is detected on entering an interactive command, RasMol
indicates the location of the error on the command line by placing the
`\verb"^"' character under the offending word or character, and writing an
error message on the following line. If a command is not recognised by
RasMol, the program will generate an `{\tt Unrecognised command!}' error
and redisplay the main prompt. If surplus information is given at the end
of a command line, RasMol will execute the recognised command, but issue
the warning message `{\tt Warning: Ignoring rest of command!}'. Some 
commands may prompt the user for more information. These commands display
a different prompt and are discussed in the command reference. 
\index{syntax error}

Whenever RasMol outputs diagnostic or error messages to the screen owing
to selecting options from the menu or picking objects on the screen, the 
current command line is cleared. And the prompt redisplayed after any text 
has been displayed.

\subsection{Command Line Editing}
\index{Command Line Editing}
\label{Editing}
RasMol allows basic editing of the command line. Pressing either backspace,
delete or \verb"^H" will delete the previous character, and the key \verb"^D"
may be used to delete the character under the cursor. Several characters
may be used to move the cursor along the command line. The characters
\verb"^B", \verb"^F", \verb"^A" and \verb"^E" move the cursor back a single
character, forward a single character, to the beginning of the line and to
the end of the line, respectively. When the cursor is not at the end of the
command line, typed characters are inserted into the line and do not 
overwrite existing characters. After a command line has been edited, a 
newline or carriage return will enter the entire line, regardless of where
the cursor is positioned. Because RasMol is unable to move the cursor up to 
the previous line, care must be taken when editing commands that wrap over
several lines. In the event that another process overwrites or corrupts the 
command line, the character \verb"^L" may be used to redisplay the line on 
the screen.
\index{Editing of commands}

RasMol maintains a history of recently used commands, so the that the user
never needs to type the same commands repeatedly. Typing \verb"^P" on the 
command line will display the previous command in the history and \verb"^N" 
will display the following command. These commands may be edited
using the features described above. Moving forward or backward through
the command history undoes the modifications made to the current line. The
number of commands retained in the history depends upon their length.
RasMol can retain more short command lines and fewer long ones.
\index{Command history}
\index{History}

For users of ``vt100'' and compatible terminals (such as an ``xterm''), RasMol
also understands the cursor control characters on the keyboard. The right 
and left cursor keys have the same affect as \verb"^F" and \verb"^B", moving
the cursor forward and back a single character respectively. Similarly, the
up and down cursor keys have the same function as \verb"^P" and \verb"^N",
producing the previous and next entries in the command history respectively.
\index{Cursor Control Characters}


\subsection{Start-up File}
\index{Initialisation File}
\index{Start-up File}

When RasMol is first run, it searches for an initialisation file of commands
to run before the command prompt is presented to the user. The file is 
called {\bf {.}rasmolrc} on UNIX systems, and {\bf RASMOL.INI} on MS-DOS
and Microsoft Windows Systems. The format and execution of the this file
is identical to that of the {\tt script} command described in 
section~\ref{CScript} of this manual. 

RasMol first looks for the initialisation file in the current directory,
and if it is not found will look for it in the user's home directory. On
MSDOS systems the environment variable `{\tt HOME}' may be used name the
appropriate directory. Unlike the command `{\tt script ".rasmolrc"}', the
program will not generate an error message if the file is not found.


\section{Menu Options}
\index{Menu options}
\label{Options}

\subsection{Main Menu}
\index{Main Menu}
\label{Main}
When RasMol is started up at the beginning of a session, the menu area
on the right hand side of the display window presents the main menu. This
is the root menu for the RasMol menu system, and this menu is redisplayed
after each operation selected from a menu is performed.
\index{Menu}

The main menu of RasMol Version 2.1 contains six options. 

\subsubsection{Load}
\index{Load submenu}
\label{MLoad}
This option allows the user to specify the Brookhaven Protein Data Bank 
(PDB) file to be read in. The function of this button is identical to the 
interactive command `{\tt load}' typed from the RasMol command line. The 
PDB file contains the Cartesian co-ordinates of each atom in the molecule 
to be displayed. A complete description of the PDB file format and records 
understood by the program (including the Raster3D colour scheme records) 
are given in section~\ref{Scheme} of this document. If a molecule has 
already been loaded from a PDB file, clicking the {\bf Load} option has no 
effect.
\index{PDB files}
\index{Load}

On pressing {\bf Load} the user is prompted by the message `{\tt PDB file 
name:}' in the terminal window. The filename must be a valid UNIX filename
without wildcard characters, and must be entered without string delimiters. 
Entering an empty filename will abort the operation and return the user to 
the main menu and RasMol command line prompt. If the file exists it is 
loaded by the program, otherwise RasMol outputs the error message `{\tt 
Error: File not found!}', before returning the user to the main menu and 
the RasMol command line prompt.

Once the PDB file is loaded, the RasMol program determines the connectivity
of the molecule by proximity of atoms: Two atoms within an appropriate
interval of one another are considered to be bonded. Once these calculations
have been performed, the program outputs statistics on the molecule together
with any appropriate information found in the PDB file. Once all the 
required preprocessing has been performed, RasMol `selects' all the atoms in
the molecule and displays a monochrome depth-cued wireframe representation
on the display window canvas. The exact details of the molecule preprocessing
steps are discussed under the `{\tt load}' command, section~\ref{CLoad} in 
the command reference section of this document.
\index{Connectivity!of atoms}
\index{Atom!connectivity}
\index{Connectivity}


\subsubsection{Display}
\index{Display submenu}
\label{MDisplay}
Selecting the {\bf Display} menu button from the RasMol main menu presents
the user with the RasMol display options submenu. This submenu allows the
current representation of the molecule to be changed. The operation of the
options on this submenu is described in detail in section~\ref{ODisplay},
``Display Menu''. These options allow the molecule to be represented as 
wireframe, alpha carbon backbone, stick bonds, space filling union-of-spheres 
or ball-and-stick models.

\subsubsection{Colours}
\index{Colours submenu}
\label{MColours}
Clicking the {\bf Colours} option on the RasMol main menu displays the
colour submenu in the display window. These submenu options allow the user
to modify the colour scheme of the atoms and bonds in the currently 
displayed molecule. The function of the buttons on the colour submenu are
described in detail in section~\ref{OColours}, ``Colours Menu''. The options 
allow the molecule to be shown in monochrome or coloured according to atom 
type, amino acid type, residue number, polypeptide chain, temperature factor 
or a user-defined colouring scheme supplied in the molecule's PDB file.

\subsubsection{Options}
\index{Options submenu}
\label{MOptions}
The {\bf Options} menu is used to access the RasMol miscellaneous options
submenu. This submenu allows the user to modify the state of the regularly
used RasMol variables. Details of each of the options available via the
RasMol options menu are described in depth in section~\ref{OOptions}, 
``Options Menu''. These options are used to turn slabbing mode, shadowing
and specular highlights on and off. In addition, the inclusion/exclusion of
hydrogren atoms and heterogenous groups from the display may be altered by
buttons on this submenu.
\index{Miscellaneous options submenu}

\subsubsection{Save}
\index{Save submenu}
\label{MSave}
The {\bf Save} menu option on the RasMol main menu is for writing the
currently displayed image to an output raster file. The functionality of
this menu option is also available through the commands `{\tt write~gif}',
`{\tt write epsf}', `{\tt write~ppm}' and `{\tt write~sunrle}'. Selecting 
the {\bf Save} button displays a submenu of raster file formats in which
the current image may be stored. This menu has the five options {\bf GIF},
{\bf EPSF}, {\bf PPM}, {\bf Rast} and {\bf Cancel}. The last option allows
the user to abort writing the current image to a file, and return to the 
main menu. The first four options specify the file format to be used in 
writing the current frame, which correspond to compuserve GIF format, 
Encapsulated PostScript, raw portable pixmap format and run length encoded 
SUN rasterfile format. 
\index{Write}
\index{GIF files}
\index{SUN rasterfiles}
\index{Encapsulated PostScript files}
\index{Portable Pixmap files}
\index{PPM files}
\index{Output raster file}
\index{Raster files}

Selecting any one of these three format options causes RasMol to prompt 
the user `{\tt Output file name:}'. The filename must be a valid UNIX 
filename without wildcard characters and must be entered without string
delimiters. Entering a empty filename causes the program to abort the save
operation and redisplay the main menu and RasMol command prompt. 
Conventionally, GIF images have the filename extension `.gif', PostScript
files have the extension `.ps', raw PPM files have the extension `.ppm' 
and RLE SUN rasterfile images the extension `.ras'. If the program is unable 
to create the given filename for writing, RasMol aborts the command with 
the error message `{\tt Output Error: Unable to create file `$\cdots$'!}'. 
Once the file has been written (which may take a short time) the main menu 
and command line prompt are automatically redisplayed.
\nocite{Adobe85a,Adobe85b}
\index{PostScript files}


If the 8-bit version of RasMol version 2.1 is running, all four file formats
may be output; however, the 24-bit versions of RasMol 2.1 may only output 
EPSF, PPM and SUN rasterfile formats. If the user attempts to write out a 
24-bit GIF image, the program warns the user with the message `{\tt Output 
Error: 24 bit GIF files unsupported!}'. Although the portable pixmap format 
represents raster images with three bytes per pixel, the eight bit version 
of RasMol produces a maximum of 250 unique RGB triples.
\index{24-bit version}
\index{8-bit version}


\subsubsection{Quit}
\index{Quit}
\label{MQuit}
The bottom button on the RasMol main menu is {\bf Quit}. The function of
this button is identical to that of the commands `{\tt quit}' and `{\tt exit}'
from the command line. {\bf Quit} terminates the RasMol program and 
returns the user to the UNIX prompt. When exiting, RasMol closes the 
display window and the image currently being displayed is lost. If a dials
box is attached to the workstation, all the LED labels above each knob
are turned off.



\subsection{Display Menu}
\index{Display submenu}
\label{ODisplay}
The display submenu allows the current representation of the molecule to 
be changed. This submenu is displayed by selecting the option {\bf Display}
on the RasMol main menu. The display submenu contains seven menu options.
The first six change the representation of the current active zone (see
section~\ref{CSelect} on the {\tt select} command) and the final option,
{\bf Cancel}, is used to return the user to the main menu without modifying
the current display.

\subsubsection{Wireframe}
\index{Wireframe}
\label{OWireframe}
This menu option displays the current active zone of the molecule as a
depth-cued wireframe model. Each bond is represented as a narrow line 
between bonded atoms. Bonds nearer the viewer are brighter than those
towards the back of the molecule, providing an impression of depth. The
colour of a bond, or {\tt none}, may be set with the {\tt colour bond}
command (section~\ref{CColour}). If the bond has no colour of its own, it 
is represented as two half bond segments, each of which is given the colour 
of the nearest bonded atom. This menu option is equivalent to the RasMol 
commands {\tt spacefill off}, {\tt backbone off} and {\tt wireframe on}.
\index{Depth-cueing}

\subsubsection{Backbone}
\index{Backbone}
\label{OBackbone}
The {\bf Backbone} menu option displays the current active zone of the
molecule as a set of cylinders connecting the alpha carbons along the
polypeptide backbone of the protein. Each `bond' is drawn as a 0.48\AA\
diameter cylinder. The colour of the backbone may be set using the RasMol
{\tt colour backbone} command (section~\ref{CColour}). If the colour of
the backbone segment is set to {\tt none}, the two halves of the cylinder
are drawn the colours of the alpha carbon atoms that they join. This 
representation is especially useful for revealing the secondary structure and
folding of a protein. This menu option is equivalent to the RasMol command 
sequence {\tt spacefill off}, {\tt wireframe off} and {\tt backbone 80}.


\subsubsection{Sticks}
\index{Sticks}
\label{OSticks}
The {\bf Sticks} menu option displays the current active zone of the 
molecule as a set of sticks model. Each bond is represented as a cylinder
of 0.48\AA\ diameter. The colour of a bond, or {\tt none}, may be altered 
with the {\tt colour bond} command (section~\ref{CColour}). If the bond 
has no colour of its own, it is represented as two half bond cylinders, 
each of which is given the colour of the nearest bonded atom. This menu 
option is equivalent to the two RasMol commands {\tt spacefill off} and 
{\tt backbone off}, followed by the command {\tt wireframe 80}.

\subsubsection{Spacefill}
\index{Spacefill}
\label{OSpacefill}
This command represents the current active zone of the molecule as a 
union-of-spheres surface. Each atom is represented as a sphere of the
appropriate van der Waals radius centred at the atom's position. The
colours of the atoms may be changed by either the {\bf Colours} submenu
(section~\ref{OColours}) or the RasMol {\tt colour atom} command 
(section~\ref{CColour}). This menu option is equivalent to the RasMol 
commands {\tt spacefill} and {\tt wireframe off}, {\tt backbone off}.
\index{Van der Waals radius}
\index{Union-of-spheres}

\subsubsection{Ball \& Stick}
\index{Ball and Stick@Ball \& Stick}
\label{OBallStick}
The RasMol {\bf Ball \& Stick} menu option draws the current active zone of 
the molecule as a ball-and-stick model.  Each atom is represented as a 
sphere of 0.48\AA\ radius, and each bond is represented as a narrow cylinder 
of 0.16\AA\ radius. This representation is combination of both a spacefilling 
and stick bond models. The colours of the atoms and bonds may be altered with
the {\tt colour atom} and {\tt colour bond} commands respectively. This menu 
option is equivalent to the four RasMol commands {\tt spacefill 120}, 
{\tt wireframe 40}, {\tt backbone off} and {\tt ribbons off}.

\subsubsection{Ribbons}
\index{Ribbons}
\label{ORibbons}
The RasMol {\bf Ribbons} menu option draws the current active zone of the
molecule as a ribbon model. This menu option is equivalent to the four
RasMol commands {ribbons 380}, {\tt spacefill off}, {\tt wireframe off}
and {backbone off}.


\subsection{Colours Menu}
\index{Colours submenu}
\label{OColours}
The colours submenu options allow the user to modify the colour scheme of 
the atoms and bonds in the currently displayed molecule. This submenu is
displayed by selecting the option {\bf Colours} on the RasMol main menu.
The colours submenu contains eight menu options. The first seven select a
colour scheme for the currently active zone (see section~\ref{CSelect} on
the {\tt select} command) and the final option, {\bf Cancel}, is used to 
return the user to the main menu without modifying the current colouring.

\subsubsection{Mono}
\index{Mono}
\label{OMono}
The top menu button on the RasMol colours menu, {\bf Mono}, is used to
colour the current active zone of the molecule monochrome white. This is
the default colour scheme on loading a PDB file using the {\tt load}
command. This menu option is equivalent to the command `{\tt colour white}'.

\subsubsection{CPK}
\index{CPK colours}
\label{OCPK}
The {\bf CPK} option on the colours submenu is used to shade the current
active zone of the molecule by atom type, using the Corey, Pauling and 
Kultun (CPK) colouring scheme. These are the colours commonly used by
chemists to denote atom type. In this scheme, carbon appears light grey,
hydrogen appears white, oxygen appears red, sulphur appears yellow, 
nitrogen appears light blue and phosphorous appears pink. Any unrecognised 
element is displayed in magenta. This menu option is equivalent to the 
command `{\tt colour cpk}'.
\index{Corey, Pauling and Kultun colours|see{CPK colours}}
\index{Colour!by atom type|see{CPK colours}}


\subsubsection{Shapely}
\index{Shapely colours}
\label{OShapely}
The {\bf Shapely} menu button on the colours submenu, changes the colour of
each atom in the current active zone to a shade determined by the atom's
residue type. For nucleic acids, each base is given a unique colour;
for proteins, each amino acid sidechain is given a unique colour, and the
the atoms of the polypeptide backbone are drawn in light grey. This is the 
colour scheme used by ``Shapely models''. This menu option is equivalent 
to the command `{\tt colour shapely}'.
\index{Colour!by residue|see{Shapely colours}}

\subsubsection{Group}
\index{Group colouring}
\label{OGroup}
The menu button {\bf Group} on the RasMol colours submenu colours each
atom of the current active zone in a colour based on its position along
the macromolecule chain it belongs to. The actual colours use a `ramped'
colour map, beginning with dark blue at the start of each chain then slowly
changing through cyan, green and yellow to red at the opposite end of
the chain. 
\index{Colour!by position along chain}

The matching of atom to colour is dependent upon the current status of the 
RasMol {\tt hetero} variable (see the {\tt set hetero} command). When the
{\tt hetero} variable is reset, and heterogenous groups are 
excluded from the default active zone, the colours are based directly on
their position along the chain. Hence the N-terminal residue of proteins 
and the $5'$-terminus of nucleic acids are coloured blue, and the C-terminal
of proteins and $3'$-terminus of nucleic acids are coloured red. If the 
{\tt hetero} variable is set, and HETATM atoms are included in the default 
active zone, each heterogenous molecule associated with a chain is coloured
as if it were appended to the end of the chain. Hence the last heterogenous 
molecule in the PDB file for each chain is drawn in red, and all atoms are
coloured based on residue number.
\index{Hetero variable}
\index{HETATM|see{Heterogenous atoms}}

Currently the functionality of this menu option is not replicated in the
RasMol command line interface.

\subsubsection{Chain}
\index{Chain colouring}
\label{OChain}
The {\bf Chain} menu option on the colours submenu colours each atom in
the current active zone using a colour based on the macromolecular chain 
to which it belongs, or with which it is assocaited. The chains are 
coloured arbitrarily by choosing the appropriate number of colours from 
a blue-to-red colour map as described in section~\ref{OGroup} above.
\index{Colour!by macromolecule chain}

Currently the functionality of this menu option is not replicated in the
RasMol command line interface.

\subsubsection{Temperature}
\index{Temperature colouring}
\label{OTemp}
The colours menu option {\bf Temperature} is used to colour the currently
active atoms of a molecule in a colour based on their anisotropic temperature
factors, as defined in the PDB file from which it was read. The actual 
colours use a `ramped' colour map, beginning with the `coldest' atoms drawn 
in dark blue, then slowly changing through the colours cyan, green and yellow 
to red, as the atoms get warmer, with red representing the `hottest' atoms 
in the file.
\index{Colour!by temperature factor}
\index{Anisotropic temperature factors}
\index{Temperature factors}

The matching of atom to colour depends upon the current value of the
RasMol {\tt hetero} variable (see section~\ref{OHetero} and 
section~\ref{CSet} on the {\tt set hetero} command). If the heterogenous 
atoms are currently excluded from the default active zone, i.e. the {\tt 
hetero} variable has the value {\tt false}, the temperature of the red 
and blue atoms are determined by the maximum and minimum temperature 
factors of the atoms in the main molecule. Otherwise, if heterogenous 
groups are included in the default active zone because the {\tt hetero} 
variable has the value {\tt true}, then the colours are taken from the 
temperature range of all the atoms in the PDB file.
\index{Hetero variable}

Currently the functionality of this menu option is not replicated in the
RasMol command line interface.

\subsubsection{User}
\index{User colouring}
\label{OUser}
The last colouring option on the RasMol colours submenu, {\bf User}, allows
the current active zone of the molecule to be coloured according to a 
user-defined colour scheme. This colour scheme is provided by including
additional `COLO' records in the PDB file containing the molecule. These
colour records, which are not supported by Brookhaven National Laboratories,
are based upon those used by David Bacon's Raster3D program~\cite{Bacon88b}
and specify the colours of individual atoms by using a pattern matching 
algorithm. The exact format of these supplemental records is described in 
detail in section~\ref{Scheme} of this document.
\index{Colour!by user schema}
\index{Raster3D}

This menu option is equivalent to the command `{\tt colour user}'.


\subsection{Options Menu}
\index{Options submenu}
\label{OOptions}

The options submenu allows user to alter commonly used display options and
parameters used in rendering molecules. This menu option is presented by
clicking the {\bf Options} menu button on the RasMol main menu. The options
submenu displays six menu options. The first five are used to toggle the
values of parameters and the final option, {\bf Cancel}, may be used to
exit the options submenu without modifying the current state of the program.

\subsubsection{Slab}
\index{Slab!option}
\label{OSlab}
The {\bf Slab} menu option on the RasMol options submenu is used to toggle
the front z-clipping function, slabbing. When slabbing is enabled, RasMol
only renders those objects (atoms and bonds) that are greater than a given
distance from the viewer. This allows the user to `cut away' the front part
of the molecule to reveal internal or occluded detail. All wireframe bonds
that intersect the clipping plane are subdivided, such that only the portion
behind the z-clipping plane is drawn. The representation of spheres that
intersect the clipping plane is governed by the {\tt slabmode} variable
(see section~\ref{CSet} on the {\tt set~slabmode} command). By default, 
{\tt slabmode} has the value {\tt solid} which depicts clipped atoms as 
solid filled spheres.
\index{Slabmode variable}
\index{Z-clipping}

When RasMol is first started, the position of the z-clipping mode is 50\% 
of the way through the molecule (drawing only the furthest half of the
structure), with slabbing switched off. The current position of the z-clipping
plane may be altered by executing a {\tt slab} command, or by using the
``SLAB'' dial on the dials box (if attached). The position of the slabbing
plane is moved by turning the ``SLAB'' dial even when slabbing is disabled.

This menu button works as a toggle button, alternatively turning on and off
slabbing depending on the current setting. The user should avoid setting the
position of the slab plane to the value 100, as the operation of the {\bf 
Slab} will appear ineffective, since the whole molecule remains displayed.
Similarly, because there is a performance degradation associated with 
rendering z-clipped images, it is recommended that slabbing is switched off, 
rather than moving the slabbing plane in front of the molecule. Shadowing may
not be performed while slabbing is enabled, so shadowing is automatically 
disabled when slabbing is switched on. This menu option performs the same 
task as the two RasMol commands {\tt slab~on} and {\tt slab~off}.

\subsubsection{Hydrogen}
\index{Hydrogen option}
\label{OHydrogen}
The {\bf Hydrogen} option on the options menu allows the user to enable
and disable the display of hydrogen atoms from the menu interface. Normally
hydrogen atoms are not resolved by X-ray crystallography and only appear
in data files generated by energy minimization programs. In order to enable
the underlying structure of the molecule to be visualised, it is often
necessary to disable the large number of hydrogen atoms that obscure the 
molecule. This is done by changing the current value of the {\tt hydrogen}
variable and selecting or restricting the current active zone as required.
By default, the {\tt hydrogen} variable has the value {\tt true} and all
hydrogen atoms and bonds are displayed along with the rest of the model.
The default active zone, as defined by the {\tt select} and {\tt restrict}
commands without parameters, relies on the current setting of {\tt hydrogen}.
\index{Hydrogen variable}
\index{Atom!hydrogen}

If {\bf Hydrogen} is clicked when RasMol is currently permitting the
display of hydrogen atoms, then the {\tt hydrogen} variable is assigned the
value {\tt false}, the {\tt restrict} command is used to force only
the default active zone to be rendered and the message `{\tt Hydrogens
removed!}' is output to the terminal window. The {\tt restrict} command
causes all objects (atoms and bonds) not in the default active zone to be
removed from the display. In normal operation, this means that all the
hydrogen atoms/bonds disappear from the screen. If the {\tt hetero} flag
is currently {\tt false}, any displayed heterogenous groups will also 
disappear from the display. This has the same effect as the command 
sequence {\tt set~hydrogen~off} followed by the command {\tt restrict}.

If the RasMol {\tt hydrogen} variable is {\tt false} when the {\bf Hydrogen}
menu option is clicked, then the {\tt hydrogen} flag is set to {\tt true}, 
the {\tt select} command is used to set the current active zone to the new
default active zone and the message `{\tt Hydrogens selected!}' is written
to the terminal window. The actual image drawn in the display window canvas
does not change, but all subsequent representation and colouring commands
will affect the molecule's hydrogen atoms. If the {\tt hetero} variable is 
currently true, all heterogenous atoms will also be `selected'. This has 
the same effect as the command sequence {\tt set~hydrogen~true} followed by 
the command {\tt select}.

\subsubsection{Het Atoms}
\index{Het Atoms option}
\label{OHetero}
The option menu {\bf Het Atoms} button allows the user to enable and disable
the display of heterogenous groups from the menu interface. Typically 
water atoms and cofactors are also resolved by X-ray crystallography, along
with the atoms of the protein or nucleic acid. Sometimes these heterogenous
groups are important in the functioning of the macromolecule, but more 
generally they are solvent moieties that co-crystallize with the main
molecule. The {\bf Het Atoms} option allows the user to define whether
these atoms, contained as HETATM records in the PDB file, should currently be
displayed. This is done by changing the current value of the {\tt hetero}
variable and selecting or restricting the current active zone as required.
By default, the {\tt hetero} variable has the value {\tt true}, and all
heterogenous atoms and bonds are displayed along with the rest of the model.
The default active zone, as defined by the {\tt select} and {\tt restrict}
commands without parameters, relies on the current setting of {\tt hetero}.
\index{Cofactors|see{Heterogenous atoms}}
\index{Atom!heterogenous}
\index{Heterogenous atoms}
\index{Select}
\index{Restrict}

If {\bf Het Atoms} is clicked when RasMol is currently permitting the
display of heterogenous atoms, then the {\tt hetero} variable is reset to the
value {\tt false}, the {\tt restrict} command is used to limit the display
of objects to the default active zone and the message `{\tt HETA atoms
selected!}' is printed below the RasMol prompt. The {\tt restrict} command
causes all objects (atoms and bonds) not in the default active zone to be
removed from the display.  In normal operation, this means all the
heterogenous atoms/bonds disappear from the screen. If the {\tt hydrogen} 
variable is currently {\tt false}, any displayed hydrogen atoms will also 
disappear from the display. This has the same effect as the command 
sequence {\tt set~hetero~off} followed by the command {\tt restrict}.

If the RasMol {\tt hetero} variable is {\tt false} when the {\bf Het Atoms}
menu option is clicked, then the {\tt hetero} flag is assigned {\tt true},
the {\tt select} command is used to set the current active zone to the new
default active zone and the `{\tt HETA atoms selected!}' diagnostic is 
displayed in the terminal window. The actual image drawn on the display 
window canvas does not change, but all subsequent representation and 
colouring commands will affect the molecule and the associated hetero
groups. If the RasMol {\tt hydrogen} variable is set, all hydrogen atoms 
will also be `selected'. This menu option is identical to the command 
sequence {\tt set~hetero~true} followed by the command {\tt select}.

\subsubsection{Specular}
\index{Specular option}
\label{OSpecular}
This menu option performs the same task as the two RasMol commands {\tt
set specular on} and {\tt set specular off}.
\index{Highlights}

\subsubsection{Shadow}
\index{Shadow option}
\label{OShadow}
The {\bf Shadow} button on the RasMol options menu toggles the value of 
the {\tt shadow} variable. When this parameter has the value {\tt true},
the macromolecular structure is rendered with cast shadows from a single
light source, otherwise the shadowing is not calculated. Although by
comparison RasMol is extremely fast at rendering shadowed images, testing
which parts of a molecule are occluded from the light source is
computationally expensive. It is recommended that, on slower machines, 
the correct orientation of the molecule is performed using the normal 
lighting model and then, once the desired position is determined, shadowing 
is enabled.

Slabbing may not be performed while shadowing is enabled, and so slabbing
is automatically disabled when shadowing is switched on.

This menu option performs the same task as the two RasMol commands {\tt
set shadow on} and {\tt set shadow off}.



\section{Atom Expressions}
\index{Atom expressions}
\label{Expressions}
\index{Expressions}

\subsection{Primitive Expressions}
\index{Primitive expressions}
\label{Primitives}
The main constituent of an atom expression is the primitive expression. A
primitive expression uniquely identifies a related group of atoms within a
molecule. A primitive expression consists of four major fields, only the
first of which is compulsory. The first field specifies the residue type,
the second specifies the residue number, the third identifies the chain (if
applicable) and the final field identifies the atoms within the residue.
A primitive expression may not contain any whitespace. Each of the fields
is described in detail below.
\index{Residue number}
\index{Residue type}
\index{Chain}


Syntax: \quad \verb"<residue-type>{<residue-number>}{<chain>}{.<atom-type>}"

The first field consists of one to three alphabetic or question mark
(`{\tt ?}') characters that identify the type of residue to be selected.
The characters are matched against the residue name of each atom in the
molecule, with the question mark acting as a wildcard character that
matches any letter. For proteins, this means that the three letter codes
for amino acids must be used. This field is case insensitive with all 
characters being converted to uppercase before the comparison is performed.
A single asterisk (`{\tt *}') character may be used as a abbreviation for 
the field ``{\tt ???}''. It is recommended that the asterisk is used 
whenever possible, for efficiency reasons. Unfortunately, residue names
containing digits, such as ``{\tt SO4}'', cannot be specified.
\index{Wildcards}

The second field contains a number that identifies the residue number
within the molecule. This field may be any number of numeric characters, or
a single asterisk to match all residues. Normally, the first field contains
an asterisk if this field is given. A numeric value in this field uniquely
identifies a single residue (base or amino acid) in the macromolecule, and 
hence a residue type given in addition would be redundant.

The third field is a single alphabetic character used to identify a chain
of the molecule (if appropriate), or either an asterisk or a question mark
to indicate that all chains should be used in the match. If a character is
specified, this is converted to uppercase before any comparison. If the
residue number field is an asterisk, a digit may be used in this field
if the chains are denoted by numerals.

The final field is a period or full-stop (`{\tt .}') character followed by
one to four alphabetic or question mark characters that identify the atom
types within a residue. All alphabetic characters are converted to capitals,
and an asterisk character may be used in place of four question mark 
characters. Because spaces are not permitted in atom types, care must be 
taken in distinguishing ambiguities, such as {\tt *.CA} meaning all alpha 
carbons or calcium atoms.


All fields that are specified in a primitive expression have to match
a given atom in order for that atom to be selected. The flexabilty permitted
by this kind of pattern-based template, allows the user to specify areas 
of interest in a molecule very quickly.

Examples:
\begin{quote}\begin{tabular}{ll}
 {\tt *}        & All atoms                             \\
 {\tt CYS}      & Atoms in cysteines                    \\
 {\tt AS?}      & Atoms in asparagine and aspartic acid \\
 {\tt *120}     & Atoms in residue 120                  \\
 {\tt *P}       & Atoms in chain P                      \\
 {\tt *.N? }    & Nitrogen atoms                        \\
 {\tt CYS.S?}   & Sulphur atoms in cysteines            \\
 {\tt SER70.C?} & Carbon atoms in serine-70 
\end{tabular}\end{quote}


\subsection{Predefined Sets}
\index{Predefined Sets}
\label{Sets}
In addition to the primitive expression, RasMol atom expressions may also
contain predefined sets. These sets are single keywords that represent
portions of the molecule of interest. Using predefined sets is a way of
abbreviating primitive expressions, and in some cases of selecting areas
of a molecule that would otherwise be difficult to distinguish. Each of
the currently implemented predefined sets is described below.
\index{Sets}

\begin{description}
\item[acidic]
The set of acidic amino acids. These are the residues Asp, Glu and Tyr.

\item[alpha]
This set contains all the $\alpha$-carbons in the protein molecule.

\item[amino]
This set contains all the atoms contained in amino acid residues. This
is useful for distinguishing the protein and DNA subunits in the current
molecule database.
\index{Amino set}

\item[backbone]
This set contains the four atoms of each amino acid that form the 
polypeptide N-C-C-O backbone for proteins, or the sugar-phosphate backbone
of nucleic acids.
\index{Backbone set}

\item[basic]
The set of basic amino acids. These are the residues Lys, Arg and His.
\index{Basic amino acids}
\index{Amino acids!basic}
\index{Basic set}

\item[cystine]
Once the disulphide bridges of a protein have been determined (using the
RasMol {\tt ssbond} command) this set contains all the atoms of cysteine
groups that are disulphide bonded, i.e. half cystines. The set off free
cysteines may be found using the expression `{\tt cys and not cystine}'.
\index{Disulphide bridges}
\index{Cysteine!free}
\index{Cysteine!half}
\index{Cystine set}
\index{SSBond}

\item[hetero]
This set contains all the heterogenous atoms in the molecule. These are all
the atoms specified by HETATM entries in the PDB file. These typically 
contain cofactors and other ligands.
\index{Atom!heterogenous}
\index{Heterogenous atoms}
\index{Hetero set}
\index{PDB files}

\item[hydrogen]
This predefined set contains all the hydrogen and deuterium atoms in the 
molecule. 
\index{Hydrogen set}

\item[hydrophobic]
This set contains all the hydrophobic amino acids. These are the amino acids
Ala, Leu, Val, Ile, Pro, Phe, Met and Trp.
\index{Hydrophobic amino acids}
\index{Amino acids!hydrophobic}
\index{Hydrophobic set}

\item[neutral]
The set of neutral amino acids. These are all the residues that are
{\tt amino} but are neither {\tt acidic} nor {\tt basic}.
\index{Neutral amino acids}
\index{Amino acids!neutral}
\index{Neutral set}

\item[polar]
The set of polar amino acids. These are all the residues that are in the
set {\tt amino} but not {\tt hydrophobic}.
\index{Polar amino acids}
\index{Amino acids!polar}
\index{Polar set}

\item[purine]
The set of purine ribonucleotides. These are the bases adenosine and
guanosine (A and G respectively).
\index{Purine set}

\item[pyrimidine]
The set of pyrimidine ribonucleotides. These are the bases cytidine and
thymidine (C and T respectively).
\index{Pyrimidine set}

\item[selected]
This set contains the set of atoms that are members of the current active
zone. These are defined by the previous {\tt select} or {\tt restrict}
commands, and not by any part of the given expression.
\index{Selected set}
\index{Restrict}
\index{Select}


\item[sidechain]
The functional side chain of the amino acid. This set contains the atoms
not in the N-C-C-O polypeptide backbone of proteins or the sugar-phosphate
backbone of nucleic acids. This predefined set is equivalent to the RasMol
expression {\tt (not backbone)}.
\index{Sidechain set}

\end{description}

\newcommand{\tic}{$\bullet$}
\begin{landscape}
\begin{table}
\begin{center}
\begin{tabular}{|l|c|c|c|c|c|c|c|c|c|c|c|c|c|c|c|c|c|c|c|c|}\hline
& \tiny ALA & \tiny ARG & \tiny ASN & \tiny ASP & \tiny CYS
& \tiny GLU & \tiny GLN & \tiny GLY & \tiny HIS & \tiny ILE
& \tiny LEU & \tiny LYS & \tiny MET & \tiny PHE & \tiny PRO
& \tiny SER & \tiny THR & \tiny TRP & \tiny TYR & \tiny VAL \\ 
Predefined Set & \bf A & \bf R & \bf N & \bf D & \bf C 
               & \bf E & \bf Q & \bf G & \bf H & \bf I 
               & \bf L & \bf K & \bf M & \bf F & \bf P 
               & \bf S & \bf T & \bf W & \bf Y & \bf V \\ \hline \hline
\tt acidic      &      &      &      & \tic &      
                & \tic &      &      &      &
                &      &      &      &      &     
                &      &      &      &      &      \\ \hline
\tt acyclic     & \tic & \tic & \tic & \tic & \tic 
                & \tic & \tic & \tic &      & \tic
                & \tic & \tic & \tic &      &     
                & \tic & \tic &      &      & \tic \\ \hline
\tt aliphatic   & \tic &      &      &      &      
                &      &      & \tic &      & \tic
                & \tic &      &      &      &      
                &      &      &      &      & \tic \\ \hline
\tt aromatic    &      &      &      &      &      
                &      &      &      & \tic &
                &      &      &      & \tic &     
                &      &      & \tic & \tic &      \\ \hline
\tt basic       &      & \tic &      &      &      
                &      &      &      & \tic &
                &      & \tic &      &      &     
                &      &      &      &      &      \\ \hline
\tt buried      & \tic &      &      &      & \tic 
                &      &      &      &      & \tic
                & \tic &      & \tic & \tic &     
                &      &      & \tic &      & \tic \\ \hline
\tt charged     &      & \tic &      & \tic &      
                & \tic &      &      & \tic &
                &      & \tic &      &      &     
                &      &      &      &      &      \\ \hline
\tt cyclic      &      &      &      &      &      
                &      &      &      & \tic &
                &      &      &      & \tic & \tic
                &      &      & \tic & \tic &      \\ \hline
\tt hydrophobic & \tic &      &      &      &      
                &      &      & \tic &      & \tic
                & \tic &      & \tic & \tic & \tic
                &      &      & \tic & \tic & \tic \\ \hline
\tt large       &      & \tic &      &      &      
                & \tic & \tic &      & \tic & \tic
                & \tic & \tic & \tic & \tic &     
                &      &      & \tic & \tic &      \\ \hline
\tt medium      &      &      & \tic & \tic & \tic 
                &      &      &      &      &
                &      &      &      &      & \tic
                &      & \tic &      &      & \tic \\ \hline
\tt negative    &      &      &      & \tic &      
                & \tic &      &      &      &
                &      &      &      &      &     
                &      &      &      &      &      \\ \hline
\tt neutral     & \tic &      & \tic &      & \tic 
                &      & \tic & \tic & \tic & \tic
                & \tic &      & \tic & \tic & \tic
                & \tic & \tic & \tic & \tic & \tic \\ \hline
\tt polar       &      & \tic & \tic & \tic & \tic 
                & \tic & \tic &      & \tic &
                &      & \tic &      &      &     
                & \tic & \tic &      &      &      \\ \hline
\tt positive    &      & \tic &      &      &      
                &      &      &      & \tic &
                &      & \tic &      &      &     
                &      &      &      &      &      \\ \hline
\tt small       & \tic &      &      &      &      
                &      &      & \tic &      &
                &      &      &      &      &     
                & \tic &      &      &      &      \\ \hline
\tt surface     &      & \tic & \tic & \tic &      
                & \tic & \tic & \tic & \tic &
                &      & \tic &      &      & \tic
                & \tic & \tic &      & \tic &      \\ \hline
\end{tabular}
\end{center}
\caption{RasMol Amino Acid Classification}
\end{table}
\end{landscape}



\subsection{Comparison Operators}
\index{Comparison operators}
\label{Operators}
Different portions of the molecule may also be described by the use of
equality, inequality and ordering operators on some of their properties.
The format for these comparison expressions is a property name followed
by a comparison operator, and then by an numeric value.

The atom properties that may be used in RasMol atom expressions are
{\tt atomno} for the PDB atom serial number, {\tt resno} for the residue
number, {\tt radius} for the displayed radius of an atom (or zero if the
atom is not represented as a sphere) and {\tt temperature} for the 
anisotropic temperature factor in the PDB file (the value given in the
comparison is actually 100 times the value specified in the file, to
ensure the value is an integer).
\index{Atom!radius}
\index{Temperature factor}
\index{Radius property}
\index{Residue number}
\index{PDB files}


The equality operator is denoted either `\verb"="' or `\verb"=="' and 
the inequality operator as either `\verb"<>"', `\verb"!="' or `\verb"/="'. 
The ordering operators are then `\verb"<"' for less than, `\verb">"' for 
greater than, `\verb"<="' for less than or equal to, and `\verb">="' for 
greater than or equal to.


Examples:
\begin{verbatim}
        resno < 23
        temperature >= 900
        atomno == 487
\end{verbatim}


\subsection{Logical Connectives}
\index{Logical connectives}
\label{Logical}
All three of the previous forms of query may be combined using the standard
boolean connectives {\bf and}, {\bf or} and {\bf not}. Both {\bf and} and
{\bf or} keywords are infix binary operators and the keyword {\bf not} is
a unary prefix operator. The order of precedence of these operators gives
logical negation the highest priority, then conjunction and finally 
disjunction has the lowest priority. Brackets, (`{\tt (}' and `{\tt )}') 
may also be used to change the order of evaluation. Both the conjunction 
and disjunction are short circuit operators, hence the ordering of 
expressions may affect the performance of queries.
\index{Not operator}
\index{And operator}
\index{Or operator}

The characters `{\tt \&}' and `{\tt |}' may be used to represent the
keywords {\tt and} and {\tt or} respectively. For notational convenience
the comma (`{\tt ,}') character may also be used for disjunction, thus
allowing a comma separated list of queries to be treated as the union of
the selected sets. The characters `\verb"~"' and `\verb"!"' may be used
to represent the keyword {\tt not}.

Examples:
\begin{verbatim}
        backbone and not alpha
        not (hydrogen or hetero)
        not *.FE and hetero
        arg, his, lys
\end{verbatim}


\section{Command Reference}
\label{Reference}

\subsection{Backbone}
\index{Backbone}
\label{CBackbone}

The RasMol {\tt backbone} command permits the representation of a polypeptide
backbone as a series of bonds connecting the adjacent alpha carbons of each 
amino acid in a chain. The display of these backbone `bonds' is turned on and 
off by the command paramater the same as the wireframe command. The command 
{\tt backbone off} turns off the selected `bonds', and {\tt backbone on} or 
with a number turns them on. The number can be used to determine the cylinder 
radius of the representation in 0.004\AA\ units. Backbone objects may be 
coloured using the RasMol {\tt colour backbone} command.
\index{Polypeptide backbone}

Syntax:
\begin{verbatim}
        backbone {<boolean>}
        backbone <value>
\end{verbatim}

The maximum permitted backbone radius is 2\AA. A parameter value of 500 or 
above results in an `{\tt Error: Integer argument too large!}' error. 

The reserved work backbone is also used as a predefined set (see section 
\ref{Sets}) and as a parameter to the `set hbond' and `set ssbond' 
commands.


\subsection{Colour}
\index{Colour}
\label{CColour}

Syntax:
\begin{verbatim}
        colour {<object>} <colour>
        color {<object>} <colour>
\end{verbatim}

The current list of objects for the colour command includes {\tt atom} for
each space filling sphere, {\tt bond} for each bond between two atoms,
{\tt backbone} the backbone `bond' between alpha carbons, {\tt hbond} for
hydrogen bonds and {\tt ssbond} for disulphide bridges. If no object 
specification is given on the command line the default object is assumed 
to be {\tt atom}.
\index{Backbone}
\index{SSBond}
\index{HBond}
\index{Atom}
\index{Bond}

The predefined colour values of the RasMol program are {\tt blue}, {\tt
black}, {\tt cyan}, {\tt green}, {\tt greenblue}, {\tt magenta}, {\tt orange},
{\tt purple}, {\tt red}, {\tt redorange}, {\tt violet}, {\tt white} and
{\tt yellow}. In additional to these predefined colours the user may specify
the exact colour as an RGB (Red-Green-Blue) triple. The syntax of an RGB
triple is a comma separated list of three numbers in square brackets. Each
of these values must lie in the range 0 to 255 inclusive and denotes the
relative amount of red, green and blue in the final colour respectively. As
an example, the following table gives the RGB equivalent to each of the
predefined colours.
\index{Colour!predefined}

\begin{center}\begin{tabular}{|c|c|} \hline
{\bf Keyword} & {\bf RGB triple} \\ \hline
{\tt blue}      & \verb"[0,0,255]"     \\
{\tt black}     & \verb"[0,0,0]"       \\
{\tt cyan}      & \verb"[0,255,255]"   \\
{\tt green}     & \verb"[0,255,0]"     \\
{\tt greenblue} & \verb"[46,139,87]"   \\ 
{\tt magenta}   & \verb"[255,0,255]"   \\
{\tt orange}    & \verb"[255,165,0]"   \\
{\tt purple}    & \verb"[160,32,240]"  \\
{\tt red}       & \verb"[255,0,0]"     \\
{\tt redorange} & \verb"[255,69,0]"    \\
{\tt violet}    & \verb"[238,130,238]" \\
{\tt white}     & \verb"[255,255,255]" \\
{\tt yellow}    & \verb"[255,255,0]"   \\ \hline
\end{tabular}\end{center}

If a colour is not recognised as being one of the RasMol predefined colours
the program consults the X Window System server colour database. If the
appropriate colour is not found by the server, RasMol displays the error
message `{\tt Error: Unknown or incorrect colour!}'.


The {\tt cpk} colouring scheme is based upon the popular plastic space
filling models, which were invented by Corey and Pauling and later 
improved by Kultun. These models consist of plastic truncated sphere
parts, which can be fitted together to form union-of-spheres surfaces
with rotatable joints. In this scheme, hydrogen appears white, oxygen 
appears red, sulphur appears yellow and nitrogen appears light blue. 
The standard molecular graphics alteration to the original colour 
scheme redefines carbon to appear as light grey instead of black. 
(Traditionally molecules are displayed on black backgrounds). The {\tt
cpk} colour scheme can not be used to colour {\tt bond} or {\tt backbone}
objects.
\index{CPK colours}

The {\tt shapely} scheme is based upon Bob Fletterick's `Shapely Models', 
and colours atoms according to the type of the residue in which they 
appear. The backbone atoms of proteins and polypeptides are rendered in 
light grey. Each of the atoms in the side chains is coloured according to
its parent amino acid. For nucleic acids, each base type is given a unique 
colour. The {\tt shapely} colour scheme can not be used to colour {\tt bond}
or {\tt backbone} objects.
\index{Shapely colours}

The {\tt none} option allows bonds, hbonds, ssbonds and backbone segments 
to be coloured according to the (alpha carbon) atoms which they connect. 
When a bond has no colour of its own, each half of its length is drawn in 
the colour of the atom to which it is connected, otherwise the `bonds' are 
drawn in the specified colour. The {\tt none} colour option can not be
used to colour {\tt atom} objects.

\subsection{HBond}
\index{HBond}
\label{CHbond}

The RasMol `{\tt hbond}' command is used to represent the hydrogen bonding of 
the protein molecule's backbone. This information is useful in assessing
the protein's secondary structure. Hydrogen bonds are represented as
dotted lines between the donor and acceptor residues. The first time that 
the `{\tt hbond}' command is used, the program searches the structure of the 
molecule to find hydrogen bonded residues and reports the number of bonds 
to the user. The command `{\tt hbond on}' displays the selected `{\tt bonds}' 
as dotted lines, and the `{\tt hbond off}' turns off their display. The colour 
of hbond objects may be changed by the `{\tt colour hbond}' command. 
Initially, the dotted line has the colours of its connected atoms.

By default the dotted lines are drawn between the accepting oxygen and 
the donating nitrogen. By using the `{\tt set hbond}' command the alpha 
carbon positions of the appropriate residues may be used instead. This is 
especially usefull when examining backbone structures.

Syntax:
\begin{verbatim}
        hbond {<boolean>}
\end{verbatim}

RasMol uses Kabsch and Sander's algorithm for determining the main chain
hydrogen bonds of proteins, as implemented by their DSSP computer program.
There is no generally corrected H bond definition, as there is no sharp
border between the quantum-mechanical and electrostatic regimes (wave 
function overlap dominates at short distances and electrostatic interaction 
at larger distnaces) and no discontinuity of the interaction energy as a
function of distance or alignment.

Main chain hydrogen bonds, used in determining secondary structure, may
be described by an electrostatic model. The electrostatic interaction
energy between two H-bonding groups is calculated by placing patial charges
on the C, O $(+q_1,-q_1)$ and N, H $(+q_2,-q_2)$ atoms. This produces the
energy equation
\[ E = q_1 q_2 ( 1/r(\mbox{ON}) + 1/r(\mbox{CH}) 
               - 1/r(\mbox{OH}) - 1/r(\mbox{CN}) ) * f \]
with $q_1 = 0.42e$ and $q_2 = 0.20e$, $e$ being the unit electron charge
and $r(\mbox{AB})$ the interatomic distance from A to B. In chemical units,
$r$ is in angstroms, the dimension factor $f = 322$, and $E$ is in kcal/mol.
Typically a good hydrogen bond has a binding energy of -3 kcal/mol.

The default cutoff value is -0.5 kcal/mol, as described in Kabsch and 
Sander's paper~\cite{Kabsch83}. This generous limit permits bifurcated H 
bonds and errors in coordinates.  

\subsection{Load}
\index{Load}
\label{CLoad}

This command allows the user to specify the Brookhaven Protein Data Bank 
(PDB) file to be read in. The PDB file contains the Cartesian co-ordinates 
of each atom in the molecule to be displayed. A complete description of the 
PDB file format and records understood by the program (including the Raster3D 
colour scheme records) is given in section~\ref{Formats} of this document. If 
a molecule has already been loaded from a PDB file, this command displays the
error message `{\tt Error: Molecule database loaded!}'.
\index{Raster3D}
\index{PDB files}

Syntax:
\begin{verbatim}
        load {"<input-file>"}
\end{verbatim}

The name of the file to load is given as a string parameter delimited by
matching single or double quotes. If a filename string is not given, the
user is prompted by the message `{\tt PDB file name:}' in the terminal 
window. The required filename may be entered without string delimiters. 
If the user enters a blank line, the {\tt load} command is aborted. The 
filename must be a valid UNIX filename without wildcard characters. If 
the file exists it is loaded by the program, otherwise RasMol outputs the 
error message `{\tt Error: File not found!}', before returning the user to 
the RasMol command line prompt.

Once the PDB file is loaded, the RasMol program determines the connectivity
of the molecule by proximity of atoms: two atoms within an appropriate
interval of one another are considered to be bonded. Currently two non-%
hydrogen atoms are considered to be bonded if they lie between 1.0\AA\ and
1.9\AA\ of one another, and an atom is considered bonded to a hydrogen if 
it is between 0.7\AA\ and 1.2\AA\ of the hydrogen's centre.
\index{Connectivity!of atoms}
\index{Atom!connectivity}
\index{Connectivity}


Once all the required preprocessing has been performed, RasMol `selects' 
(see section~\ref{CSelect}) all the atoms in the molecule and displays a 
monochrome depth-cued wireframe representation on the display window canvas. 

The RasMol {\tt load} command displays status information on the contents
of the file. An example of this output is presented below. The program
determines the molecule name, classification and Brookhaven code from the
appropriate records in the PDB file. A count of the number of chains, groups, 
atoms and bonds in the main molecule is displayed. If the data file contains 
heterogenous atoms, such as cofactors, the number of hetero atoms and groups
are not included in these totals but presented in brackets after the main
values.

\begin{verbatim}
        Molecule name ..... CRAMBIN
        Classification .... PLANT SEED PROTEIN
        Brookhaven Code ... 1CRN
        Number of Groups .. 46
        Number of Atoms ... 327
        Number of Bonds ... 379
\end{verbatim}

If the {\tt load} command is followed by anything other than a string, the
program outputs a `{\tt Error: Filename string expected!}' error message.

\subsection{Exit}
\index{Exit}
\label{CExit}
The {\tt exit} command is used to terminate a RasMol session. The behaviour
of the {\tt exit} command is identical to that of the {\tt quit} command. 
When either an {\tt exit} or a {\tt quit} command is executed from the
command line, the program closes the display window if the program is being
used with a colour X Windows display and clears the LED labels on the dials
box (if one is connected to the workstation), before terminating. Both the
{\tt exit} and {\tt quit} commands finish executing the current script file
when executed in either a script file or the `.rasmolrc' startup file. This
command may also be executed by selecting the {\bf Quit} option on the
RasMol main menu.

\subsection{Quit}
\index{Quit}
\label{CQuit}
The {\tt quit} command is used to terminate a RasMol session. The behaviour
of the {\tt quit} command is identical to that of the {\tt exit} command.
See section~\ref{CExit}.


\subsection{Restrict}
\index{Restrict}
\label{CRestrict}
The {\tt restrict} command is used to limit the display of the molecule to
a specified portion. This command is similar to the {\tt select} command
(see section~\ref{CSelect}), except that any object not in the active zone
after the command is executed is removed from the display.
\index{Active Zone}

Syntax:
\begin{verbatim}
        restrict {<expression>}
\end{verbatim}


\subsection{Rotate}
\index{Rotate}
\label{CRotate}
The {\tt rotate} command is used to rotate the molecule on the screen. The
molecule may be rotated about its centre around any of the three co-ordinate
axes X, Y and Z. The first parameter of the {\tt rotate} command specifies
which axis to rotate. The final parameter to the {\tt rotate} command is an
integer specifying the angle to be transformed (in degrees). This value may 
be prefixed by a single unary minus sign to rotate the molecule in the 
opposite sense. For positive angles, rotating about X moves the closest 
point on the molecule down, rotating about Y moves the closest point to the 
right, and rotation about Z spins the molecule clockwise. Whitespace may 
appear between the unary minus and the first digit of the value. A unary 
minus is allowed to prefix a zero value.
\index{Rotating the molecule}

If the RasMol program is being used interactively, the model may also be
rotated by the two scroll bars to the right of and below the canvas area of
the display window. The {\tt rotate} command updates the position of the
scroll bar indicators.
\index{Scroll bars}

If there is a dials box attached to the workstation, the first three dials
on the top row may also be used to rotate the molecule about the X, Y and
Z axes respectively. Turning these dials rotates the molecule relative to
its current position.

Syntax:
\begin{verbatim}
        rotate <axis> {-} <value>
\end{verbatim}

An axis parameter other than `{\tt x}', `{\tt y}' and `{\tt z}' generates
an `{\tt Invalid command syntax!}' error. A missing or non-numeric integer
value parameter causes an `{\tt Integer value expected!}' error. Extremely
large numeric values are internally truncated to 32 bit values.


\subsection{Script}
\index{Script}
\label{CScript}

The RasMol `{\tt script}' command executes a set of commands that are 
contained in a text file as if they had been typed in interactively. Each
command is placed on its own line in the text file terminated by a 
carriage return or newline character. The comment character, `{\tt \#}'
may be used to place to annotate the file of commands, as described in
section~\ref{Command}. The behaviour of the RasMol commands `{\tt exit}'
and `{\tt quit}' within a script file cause, the program to cease 
execution of that file rather than terminate the RasMol program. RasMol
{\tt script} commands may be contained within script files, allowing
command files to be nested upto a maximum of ten deep.

If any error occurs with a command within a script file, the error 
messages generated by RasMol are annotated with the name of the current
script file and the line number in that file of the incorrect command.

Syntax:
\begin{verbatim}
        script <filename>
\end{verbatim}

Omitting the filename parameter from after the {\tt script} command
produces a `{\tt Filename string expected}' error message. If RasMol is 
unable to find and open the specified script file, they program will 
display the error message `{\tt Cannot open script file '{\rm\it name}'!}', 
where {\it name} is the name of the file given on the command line.

If the command file contains any lines that contain more than 255
characters, the program issues the error message `{\tt Script command
line too long!}'. Any commands on this line are ignored and execution
of the file resumes on the following line of the file. If the nesting
of script files exceeds the maximum depth, RasMol generates the error
message `{\tt Script command stack too deep!}'.


\subsection{Select}
\index{Select}
\label{CSelect}
The {\tt select} command is used to define the active zone of a molecule. The 
active zone of a molecule is that portion of the molecule model that is
affected by commands that alter the representation and colour of atoms
and bonds. The select command does not change the image displayed on the
screen.
\index{Active Zone}

Syntax:
\begin{verbatim}
        select {<expression>}
\end{verbatim}


\subsection{Set}
\label{CSet}

The RasMol {\tt set} command allows the user to alter various internal 
parameters of the rendering process. The syntax of an {\tt set} statement
specifies a `variable' or `parameter' to be modified followed by the value
to be assigned to that parameter. The effects of each of these internal
parameters is discussed separately in the following paragraphs.

Syntax:
\begin{verbatim}
        set ambient [<value>]
        set background <colour>
        set bondzone <zonemode>
        set hetero <boolean>
        set hydrogen <boolean>
        set shadow <boolean>
        set slabmode <slabmode>
        set specular <boolean>
        set specpower <value>
\end{verbatim}
\index{Hydrogen variable}
\index{Hetero variable}
\index{Specular option}
\index{Highlights}
\index{Shadow}

\subsubsection*{Set Ambient}
The RasMol {\tt ambient} parameter is used to determine the amount of
ambient light in the scene. Ambient light is the pervasive light not
emitted from the primary light source. This is used to determine the
intensity of surfaces that face away from the light, or areas hidden in
shadow. The permisible integer values, betweem 0 and 100 inclusive, give
the ambient light as a percentage of the primary light intensity. Omitting
the argument returns the parameter to its default value.

\subsubsection*{Set Background}
The RasMol {\tt set background} command is used to change the colour of the
canvas window background. By default, this colour is initially black. The
background may be changed to any of the `generic' colour values used by the
{\tt colour} command (see section~\ref{CColour}). These colours include the
predefined colours, RGB colour triples and entries in the X Window System 
server's colour database. Non-generic colours, such as {\tt cpk}, {\tt none}
and {\tt user} may not be specified. This command is usefull in preparing
overhead slides from RasMol images that require light background colours.
\index{Background colour}

\subsubsection*{Set BondZone}
A RasMol bond zone mode is either `{\tt and}' ({\tt false}) or `{\tt or}' 
({\tt true}). An `{\tt and}'' value selects a bond if if both bonded atoms 
are currently selected, whereas an `{\tt or}' value selects a bond if either 
of the connected atoms is in the current active zone.
\index{Bond!selection}
\index{Bond zone mode}
\index{And operator}
\index{Or operator}

\subsubsection*{Set Slabmode}
The current RasMol slab modes are {\tt reject}, {\tt half}, {\tt hollow},
{\tt solid} and {\tt section}. In {\tt reject} slab mode, any atom that is 
intersected by the z-clipping plane when slabbing is enable is not displayed 
at all.  In {\tt half} slab mode, each atom is represented as the front half
of a hollow sphere. This allows atoms to be visible through the cut away
section of a sliced atom. Slab mode {\tt hollow} treats each atom as a full
hollow sphere, and sliced atoms reveal the inside surfaces of their `shell'.
Finally, the {\tt solid} slab mode represents the spheres as solid spheres:
any sphere cut by the slabbing plane reveals a solid cross section of the
interior. The colour of the cross-section is taken from the closest atom,
hence two intersecting spheres have a colour boundry through their cross
section. The {\tt section} slab mode is identical to {\tt solid} mode 
except that only the cross sections of atoms intersecting the slab plane
are drawn. This gives a two dimensional slice through the molecule.
\index{Slabmode variable}
\index{Slab!representation}
\index{Cross Section}
\index{Atom}

\subsubsection*{Set SpecPower}
The RasMol {\tt specpower} variable describes the shininess of surfaces 
in the scene when the specular highlights are enabled by the {\tt set
specular} command. The larger the integer argument the shinier (or more
reflective) the surfaces become. Low values produce large white dots on
spheres (like those seen on plastic), while larger values produce smaller
white dots (like those observed on metal or glass). Ommiting the argument
from this command returns the {\tt specpower} variable to its default 
setting.

\subsubsection*{Set Specular}
The RasMol {\tt specular} variable determines whether surfaces are to be
rendered as shiny or diffuse surfaces. When the {\tt specular} variable 
has the value {\tt true} or {\tt on}, all surfaces are considered shiny
with typically a small white dot representing the reflection of the light
source on the surface. If {\tt specular} is {\tt false} or {\tt off} all
surfaces are rendered with only diffuse lighting. The `shininess' of the
surface is determined by the value of the {\tt specpower} variable.


\subsection{Show}
\index{Show}
\label{CShow}

The RasMol `{\tt show}' command displays usefull information about the
currently loaded molecule. The `{\tt show sequence}' command lists the 
sequence of residues that form the macromolecule; amino acids for protein 
and nucleotides for nucleic acids. The `{\tt show info}' command displays
the name of the molecule, the number of groups, atoms and bonds it contains
together with some of the header information stored in the PDB file.


Syntax:
\begin{verbatim}
        show sequence
        show info
\end{verbatim}

The {\tt show sequence} command is particularly usefull for displaying the
residue type and residue numbers of a macromoecule in order to prepare an
atom expression for the required portion of the molecule. If the current
macromolecule contains more than one chain, each the internal name of each
chain is displayed on a separate line followed by the sequence that makes 
up that chain. The sequences are displayed ten residues to a line, giving 
the full residue name followed by its residue number. An example output
for the protein {\em crambin} is given below.

\begin{small}
\begin{verbatim}
THR1   THR2   CYS3   CYS4   PRO5   SER6   ILE7   VAL8   ALA9   ARG10  
SER11  ASN12  PHE13  ASN14  VAL15  CYS16  ARG17  LEU18  PRO19  GLY20  
THR21  PRO22  GLU23  ALA24  ILE25  CYS26  ALA27  THR28  TYR29  THR30  
GLY31  CYS32  ILE33  ILE34  ILE35  PRO36  GLY37  ALA38  THR39  CYS40  
PRO41  GLY42  ASP43  TYR44  ALA45  ASN46  
\end{verbatim}
\end{small}

The {\tt show information} command displays the details of the size of the
current molecule to the user. It presents data on the molecule's name, 
classification and brookhaven code (from the appropriate records in the PDB
file), the number of chains, groups, atoms and bonds. If the data file
contains any heterogenous groups, such as cofactors, these group and their
atoms are not included in the above totals, but presented in brackets after
the main values. An example output for the protein {\em crambin} is given
below.

\begin{verbatim}
        Molecule name ..... CRAMBIN
        Classification .... PLANT SEED PROTEIN
        Brookhaven Code ... 1CRN
        Number of Groups .. 46
        Number of Atoms ... 327
        Number of Bonds ... 379
\end{verbatim}

Giving an option other than {\tt sequence}, {\tt info} or {\tt information}
after the {\tt show} command results in a RasMol ``{\tt Invalid command 
argument!}'' error message.


\subsection{Slab}
\index{Slab}
\label{CSlab}

The RasMol {\tt slab} command may be used to turn z-clipping on and off,
and positioning the z-clipping plane. Slabbing involves removing all the
objects on the screen that are closer than a given distance from the viewer.
This position is referred to as the location of the (front) z-clipping plane.
When slabbing is enabled, all wireframe bonds that intersect the clipping
plane are split, and only the section of the line furthest from the viewer is
drawn. The representation of spheres that intersect the clipping plane is
governed by the {\tt slabmode} variable (see the {\tt set~slabmode} command).
By default, {\tt slabmode} is set to the value {\tt solid} which depicts 
clipped atoms as solid filled spheres.
\index{Z-clipping}

The position of the z-clipping (or slabbing) plane is represented as a 
value between 0 and 100. This represents a linear scale from the very back 
of the molecule to the front. This value may also be considered the 
percentage visibility, for example, a value of 25 typically only display 
one quarter of the atoms/bonds in the molecule. When RasMol is first started 
up, the initial value of the clipping plane is 50 (drawing only the back 
half of the molecule) and the slabbing mode is disabled.

There is a performance degradation associated with rendering images with 
slabbing enabled, hence it is recommended that slabbing is turned off rather
than the position of the z-clipping plane being set to 100 (in front of the 
molecule). Similarly, the user should be aware that a blank screen may be
caused by a low value of the slabbing plane while slabbing is enabled.

When the {\tt slab} command followed by a boolean value, RasMol enables or
disables slabbing without affecting the value of the clipping plane. Hence,
if the {\tt slab} command is followed by the keyword {\tt on} or {\tt true}
slabbing is switched on, and if it is followed by either {\tt off} or {\tt
false}, then slabbing is turned off. Executing the {\tt slab} command with 
no arguments ensures that slabbing is currently enabled.

The {\tt slab} command may also be given with a numeric parameter to specify
the exact position of the slabbing plane. The value of this parameter must
lie between 0 and 100. A slab value of 0 displays none of the molecule, a 
value of 50 displays the back half of it and a value of 100 shows all of it. 
Specifying a value to the slab command, automatically enables slabbing.

If there is a dials box connected to the workstation, the bottom right hand
dial, labelled ``SLAB'', may be used to interactively move the clipping plane
when slabbing mode is enabled. Turning the dial moves the location of the
clipping plane relative to its current setting. 


Syntax:
\begin{verbatim}
        slab {<boolean>}
        slab <value>
\end{verbatim}

An unrecognised parameter to a {\tt slab} command generates an `{\tt Invalid
command syntax!}' error message. A numeric value greater than 100 produces
an `{\tt Integer argument too large!}' error.


\subsection{Spacefill}
\index{Spacefill}
\label{CSpacefill}

The RasMol {\tt spacefill} command determines the radius of each atom in its
spherical representation on the screen. The currently selected zone is
represented space filling union-of-spheres model. An integer parameter may
be used to specify the radius of each atom, given in 4nm units. The maximum
permitted value is 500, which corresponds to a radius of 2\AA. If no 
parameter is given, the radius is taken to be the Van der Waals radius of
the appropriate atom.

The temperature option is used to set the radius of each selected sphere
to the value in the temperature field of the molecule file. A zero or
negative value causes no change in the selected atom. Temperature values
greater than 2.00 are truncated to the maximum 2\AA\ radius.

\nocite{Lesk91,Iijima87}

The user option allows the radius of the selected spheres to be determined
by matching each atom against optional lines in the input data file. Details
of the wildcard pattern matching used by Raster3D's COLOR records is given 
in section~\ref{Scheme} of this document.

Syntax:
\begin{verbatim}
        spacefill {<boolean>}
        spacefill <value>
        spacefill temperature
        spacefill user
\end{verbatim}

A numeric argument over 500 results in the RasMol error `{\tt Integer 
argument too large}'.


\subsection{SSBond}
\index{SSBond}
\label{CSSBond}

The RasMol `{\tt ssbond}' command is used to represent the disulphide bridges
of the protein molecule as dotted lines between the connected cysteines.
The first time that the `{\tt ssbond}' command is used, the program searches
the structure of the molecule to find half-cysteine pairs and reports the
number of bridges to the user. The command `{\tt ssbond on}' displays the 
selected `bonds' as dotted lines, and the `{\tt ssbond off}' turns off their
display. The colour of ssbond objects may be changed by the `{\tt colour
ssbond}' command. Initially, the dotted line has the colours of its 
connected atoms.

By default the dotted lines are drawn between the sulphur atoms within
the cysteine groups. By using the `{\tt set ssbond}' command the position of
the cysteine's alpha carbons may be used instead. This is especially
usefull when examining backbone structures.

Syntax:
\begin{verbatim}
        ssbond {<boolean>}
\end{verbatim}

RasMol determines disulphide bonds, i.e. covalent links between the 
S$^\gamma$ atoms of two Cys residues, to exist between two sulphur
atoms separated by a distance of less than 3.0\AA.


\subsection{Translate}
\index{Translate}
\label{CTranslate}

The RasMol {\tt translate} command permits the molecule to be translated
(shifted) along each of the three major axes. The integer argument specifies
an absolute position along the given axis to move. Positive values move the
molecule right and up for the `{\tt x}' and `{\tt y}' axes, and negative 
values left and down for the `{\tt x}' and `{\tt y}' axes respectively. An
absolute value of 100 moves the molecule completely off of the canvas window,
an absolute value of 50 places the center of the molecule at the very edge of
the screen and a zero parameter places the center of the molecule in the 
middle of the screen with respect to the given axis.

Syntax:
\begin{verbatim}
        translate <axis> {-} <value>
\end{verbatim}

An axis parameter other than `{\tt x}', `{\tt y}' and `{\tt z}' generates
an `{\tt Invalid command syntax!}' error. A missing or non-numeric integer
value parameter causes an `{\tt Integer value expected!}' error. Integer 
values greater than 100 or less than -100 produce an `{\tt Integer argument
too large!}' error.


\subsection{Wireframe}
\index{Wireframe}
\label{CWireframe}

The RasMol {\tt wireframe} command determines the representation and radius
of each bond on the screen.

Syntax:
\begin{verbatim}
        wireframe {<boolean>}
        wireframe <value>
\end{verbatim}


\subsection{Write}
\index{Write}
\label{CWrite}

The {\tt write} command is used to output the currently displayed image to 
a raster file. RasMol allows the format of the generated output file to
be set to one of several standard formats. This file format may be optionally 
specified by following the {\tt write} command by the keyword {\tt gif}, 
{\tt ppm}, {\tt sun}, {\tt sunrle} or {\tt ps}, which denote compuserve 
GIF format, raw portable pixmap format, standard sun rasterfile, run length 
encoded sun rasterfile and PostScript, respectively. If a file format is 
not specified on the command line, the image is written in GIF file format.

The final parameter to the write command is a string containing the filename
to which the image should be written. This string must be delimited by 
matching single or double quotation marks. If a filename is not given on the
command line, the program prompts the user for the required filename with 
the prompt `{\tt Output file name:}'. A filename typed into the terminal 
window at this prompt should not be delimited by quotation marks. Entering 
an empty filename at this prompt will cause RasMol to abort the {\tt write} 
command. The filename must be a valid UNIX filename without any wildcard 
characters. Conventionally GIF images have the filename extension `.gif', 
raw PPM files have the extension `.ppm', SUN rasterfile images the extension
`.ras' and PostScript files have the extension `.ps'.
\index{GIF files}
\index{SUN rasterfiles}
\index{Encapsulated PostScript files}
\index{Portable Pixmap files}
\index{PPM files}
\index{Output raster file}
\index{Raster files}

The resolution of the created image file is taken from the current size of
the display window canvas area. Hence different sized images may be created 
by resizing the window appropriately before executing the {\tt write} command.
If the image is being prepared non-interactively (i.e. not from a suitable
colour frame buffer), the image is generated at $576 \times 576$ resolution,
which is the default size of the canvas area of the display window.

The current 24 bit version of RasMol, version 2.1, is unable output into the
compuserve GIF file format for technical reasons. If the user attempts to 
write out a 24 bit GIF image, the program warns the user with the message 
`{\tt Output Error: 24 bit GIF files unsupported!}'. Although the portable 
pixmap format represents raster images with three bytes per pixel, the eight 
bit version of RasMol only produces a maximum of 250 unique RGB triples.

Syntax:
\begin{verbatim}
        write {<format>} {"<output-file>"}
\end{verbatim}

If the {\tt write} command is followed by an unrecognised file format, or if
the output filename is not a valid string, RasMol issues a `{\tt Filename 
string expected!}' error message. If the program is unable to write to the 
given file, RasMol aborts the command with the error message `{\tt Output 
Error: Unable to create file `$\cdots$'!}'. 


\subsection{Zoom}
\index{Zoom}
\label{CZoom}

The {\tt zoom} command is used to modify the magnification of the molecule
being displayed. The molecule is always magnified with respect to its centre.
The zoom parameter may be used to scale the molecule between
10\% and 200\% of its default magnification. The default scale of
the molecule is chosen to allow the molecule to be rotated about any angle
and still fit within the display canvas (i.e. not require clipping). There
is a performance degradation with displaying molecules that clip the edges
of the screen.
\index{Scale|see{Zoom}}

The {\tt zoom} command without any parameters or followed by the keywords
{\tt off} or {\tt false}, turns off the magnification of the molecule and
sets the scale to its initial value. The {\tt zoom} command followed by
the keyword {\tt on} or {\tt true} magnifies the molecule to 1.5 times its
initial scale.

The {\tt zoom} command may also be given with a numeric parameter to specify
the exact magnification required. The value of this parameter must lie 
between 0 and 200. This value represents the percentage magnification 
required, with a value of 100 representing the default scale of the molecule 
and a value of 200 rendering the molecule twice its normal size. Numeric 
values less than or equal to 10 generate minimum scale images at one tenth 
the size of the original. The command `{\tt zoom on}' has the same effect 
as the command `{\tt zoom 150}'.

If there is a dials box connected to the workstation, the top right hand
dial, labelled ``ZOOM'', may be used to magnify the molecule interactively.
Turning the dial alters the magnification of the molecule relative to its
current setting.


Syntax: 
\begin{verbatim}
        zoom {<boolean>}
        zoom <value>
\end{verbatim}

An unrecognised parameter to a {\tt zoom} command generates an `{\tt Invalid
command syntax!}' error message. A numeric value greater than 200 produces
an `{\tt Integer argument too large!}' error.


\section{File Formats}
\index{File formats}
\label{Formats}

\subsection{Brookhaven Data Bank Files}
\index{PDB files}
\label{Brookhaven}

If you do not have the Brookhaven documentation, you may find the following
summary of the PDB file format useful. The Protein Data Bank is a computer-%
based archival database for macromolecular structures~\cite{Bernstein77}.
The database was established in 1971 by the Brookhaven National Laboratory,
New York, as a public domain repository for resolved crystallographic
structures. The Bank uses a uniform format to store atomic co-ordinates 
and partial bond connectivities as derived from crystallographic studies. 

PDB file entries consist of records of 80 characters each. Using the
punched card analogy, columns 1 to 6 contain a record-type identifier,
the columns 7 to 70 contain data. Columns 71 to 80 are normally blank,
but may contain sequence information added by library management 
programs. The first four characters of the record identifier are 
sufficient to identify the type of record uniquely, and the syntax of
each record is independent of the order of records within any entry
for a particular macromolecule.

The only record types that are of major interest to the RasMol program
are the ATOM and HETATM records (apart from COLO record type not
sanctioned by Brookhaven), which describe the position of each
atom. ATOM/HETATM records contain standard atom names and residue
abbreviations, along with sequence identifiers, co-ordinates
in Angstrom units, occupancies and thermal motion factors. The exact
details are given below as a FORTRAN format statement.

\begin{center}
{\tt FORMAT(6A1,I5,1X,A4,A1,A3,1X,A1,I4,A1,3X,3F8.3,2F6.2,1X,I3)}
\end{center}

\begin{center}\begin{tabular}{|c|l|} \hline
{\bf Column} & \multicolumn{1}{c|}{\bf Content} \\ \hline
1-6   & `ATOM' or `HETATM' \\
7-11  & Atom serial number (may have gaps) \\
13-16 & Atom name, in IUPAC standard format \\
17    & Alternate location indicator indicated by A,B or C \\
18-20 & Residue name, in IUPAC standard format \\
23-26 & Residue sequence number (order as below) \\
27    & Code for insertions of residues (i.e. 66A \& 66B) \\
31-38 & X co-ordinate \\
39-46 & Y co-ordinate \\
47-54 & Z co-ordinate \\
55-60 & Occupancy \\
61-66 & Temperature factor \\
68-70 & Footnote number \\ \hline
\end{tabular}\end{center}

Residues occur in order of their sequence numbers, which always increase
starting from the N-terminal residue for proteins and $5'$-terminus for
nucleic acids. If the residue sequence is known, certain atom serial
numbers may be omited to allow for future insertion of any missing atoms.
Within each residue, atoms are ordered in a standard manner, starting with
the backbone (N-C-C-O for proteins) and proceeding in increasing remoteness
from the alpha carbon, along the side chain. 

HETATM records are used to define post-translational modifications and
cofactors associated with the main molecule. Optional TER records are
interpreted as breaks in the main molecule's backbone.

If present, RasMol also inspects HEADER, COMPND and END records. Information
such as the name, Brookhaven code, revision date and classification of the 
molecule are extracted from HEADER and COMPND records, and the end of the 
file may be indicated by an END record.


\subsection{PDB Colour Scheme Specification}
\index{Colour scheme format}
\label{Scheme}
The RasMol v2.1 program also accepts the supplementary record-type COLO
in the PDB file format. The record was introduced by David Bacon's 
Raster3D program for specifying the colour scheme to be used when 
rendering the molecule~\cite{Bacon88b,Bacon88a}. This extension to 
the standard file format is not currently supported by Brookhaven.
\index{Raster3D}

The COLO record has the same basic record type as the ATOM and HETATM
records described above, but with COLO in the first four columns:

\begin{center}\begin{tabular}{|c|l|} \hline
{\bf Column} & \multicolumn{1}{c|}{\bf Content} \\ \hline
1-6   & `COLO' \\
7-30  & Mask (described below) \\
31-38 & Red component \\
39-46 & Green component \\
47-54 & Blue component \\
55-60 & Sphere radius in Angstroms \\
61-80 & Comments \\ \hline
\end{tabular}\end{center}

Note that the Red, Green and Blue components are in the same positions
as the X, Y, and Z components of an ATOM or HETA record, and the van
der Waals radius goes in the place of the Occupancy. The Red, Green and
Blue components must all be in the range 0 to 1.

Colours are assigned to atoms using a matching process. The Mask field
is used in the matching process as follows. First RasMol reads in and
remembers all the ATOM, HETA and COLO records in input order. When the
user-defined (`User') colour scheme is selected, RasMol goes through each 
remembered ATOM/HETA record in turn, and searches for a COLO record 
that matches in all of columns 7 through 30. The first such COLO record
to be found determines the colour and radius of the atom.

In order that one COLO record can provide colour and radius specifications
for more than one atom (e.g. based on residue, atom type, or any other
criterion for which labels can be given somewhere in columns 7 through 
30), a `don't-care' character, the hash mark ``\#'' (pound sign, number sign,
sharp sign) is used. This character, when found in a COLO record, matches 
any character in the corresponding column in a ATOM/HETA record. All other 
characters must match identically to count as a match. As an extension to 
the specification, any atom that fails to match a COLO record is displayed 
in white.
\index{Hash character}
\index{Wildcards}


\section{Contact}
\label{Contact}
If you have any questions, problems, comments, complaints or suggestions
please do not hesitate to call or write to the author:

\begin{center}\begin{tabular}{ll}
Roger Sayle  & \\
Biocomputing Research Unit & Phone: (+44) (0)31 650 5163 \\
Room 707, Darwin Building  & EMail: \verb"ros@biocomp.ed.ac.uk" \\
University of Edinburgh & \\
Mayfield Road & \\
Edinburgh EH9 3JR & \\
\quad\quad UK &
\end{tabular}\end{center}

\nocite{Max83}

\nocite{Sayle90,Sayle92}
\bibliographystyle{plain}
\bibliography{rasmol}
\printindex
\end{document}
